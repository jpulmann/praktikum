\documentclass[a4paper, 10pt]{article}
\usepackage[left=3cm, right=3cm, bottom=3cm, top =3cm]{geometry}
\usepackage[utf8]{inputenc}
\usepackage[slovak]{babel}
\usepackage[IL2]{fontenc}
\usepackage{amsmath}
\usepackage{amsfonts}
\usepackage{amssymb}
\usepackage{graphicx}
\usepackage{color}
\usepackage{booktabs}
\usepackage{wrapfig}
\usepackage[version=3]{mhchem}


\usepackage{float}
\newfloat{graph}{h}{graphs}
\floatname{graph}{Graf}
\usepackage{caption}

\usepackage{hyperref}
\usepackage[compact]{titlesec}
\newcommand{\dd}{\ensuremath{ \mathrm{d} }}
\newcommand{\unit}[1]{\ensuremath{\, \mathrm{#1}}}
\newcommand{\di}[1]{\ensuremath{_\mathrm{#1}}}

\begin{document}
\titlespacing{\section}{0pt}{*0}{*0}
\titlespacing{\subsection}{0pt}{*0}{*0}
\titlespacing{\subsubsection}{0pt}{*0}{*0}
\title{NAZOV}
\author{Ján Pulmann}
\date{DATUM}
\maketitle
%%%
\section*{Úlohy}
\begin{enumerate}

	\item ULOHY
 \end{enumerate}
 
 %%%
\section*{Teória}
\begin{equation}
\label{eq:teor:bragg}
2 d_{hkl} \sin\theta_{hkl} = \lambda\,.
\end{equation}
%%%
\section*{Postup merania}
\subsection*{Pomôcky}
%%%
\section*{Výsledky merania}

\begin{graph}[h!]
\centering
\vspace*{-15pt}
%~ \input{output/kombinovane_spektrum.tex}
\caption{ Spektrá namerané s a bez tienenia \label{graph:spektrum}}
\end{graph}


\begin{table}[h!]
\centering
\hspace*{30pt}
\begin{tabular}{c|c|c|c}
$ 2\theta \,/\,^\circ $ & 
Intenzita $N$ &
$ d_{hkl}\,/\unit{\AA}$ & 
$ Q_i$ 
\\
\midrule 
%~ \input{output/Qi.tex}
\end{tabular}
\newline
\vspace*{2pt}
\caption{Vypočítané medzirovinné vzdialenosti a pomery $d_{hkl}$\label{tab:Qi_merane}}
\end{table}

%%%
\section*{Diskusia}

%%%
\section*{Záver}
%%%

\begin{thebibliography}{9}

\bibitem{stud}
    \emph{Stránky s pokynmi ku úlohe A19} \\
    \url{http://krystal.karlov.mff.cuni.cz/kfes/vyuka/lp/
} 18.11.2013

\bibitem{pres}
    Systematic Errors and Sample Preparation for X-Ray Powder Diffraction.  \textit{Jim Connolly}. prezentácia
    \url{http://epswww.unm.edu/xrd/xrdclass/07-Errors-Sample-Prep-PPT.pdf}


\end{thebibliography}
\end{document}
