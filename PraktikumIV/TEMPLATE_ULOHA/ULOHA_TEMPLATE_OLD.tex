\documentclass[a4paper, 10pt]{article}
\usepackage[left=3cm, right=3cm, bottom=3cm, top =3cm]{geometry}
\usepackage[utf8]{inputenc}
\usepackage[slovak]{babel}
\usepackage[IL2]{fontenc}
\usepackage{amsmath}
\usepackage{amsfonts}
\usepackage{amssymb}
\usepackage{graphicx}
\usepackage{color}
\usepackage{booktabs}
\usepackage{wrapfig}
\usepackage{caption}
\usepackage{hyperref}
\usepackage[compact]{titlesec}
\newcommand{\dd}{\ensuremath{ \mathrm{d} }}
\newcommand{\unit}[1]{\ensuremath{\, \mathrm{#1}}}
\newcommand{\di}[1]{\ensuremath{_\mathrm{#1}}}

\begin{document}
\titlespacing{\section}{0pt}{*0}{*0}
\titlespacing{\subsection}{0pt}{*0}{*0}
\titlespacing{\subsubsection}{0pt}{*0}{*0}
\title{ULOHA}
\author{Ján Pulmann}
\date{1. 3. 2013}
\maketitle
%%%
\section*{Úlohy}
\begin{enumerate}

	\item 
	\item 

 \end{enumerate}
 
 %%%
\section*{Teória}

\begin{equation}
\label{eq:}
\end{equation}


\begin{figure}[htb]
\centering
%\includegraphics[scale=1]{zap1.eps}
\caption{Zapojenie usmerňovača}
\label{fig:zap1}
\end{figure}

%%%
\section*{Postup merania}

\subsection*{Pomôcky}

%%%
\section*{Výsledky merania}

\begin{table}[h!]
\centering
\hspace*{60pt}
\begin{tabular}{r|r||r|r}
$ n $ & 
$ x\,/\unit{\mu m}$ &
$ n$ & 
$ x\,/\unit{\mu m}$ 
\\
\midrule 
%\input{data/kalib_tab.tex}
\end{tabular}
\newline
\vspace*{2pt}
\caption*{\textbf{ Tab. 1} Kalibrácia - závislosť počtu \\prekrytých maxím na posunutí meradla $x$}
\end{table}

\begin{wraptable}[25]{r|r}[10pt]{3cm}
\vspace*{-30pt}
\textbf{ Tab. 1}\newline Napätie
\newline
\begin{tabular}{r|r}
$ C\, / \unit {\mu F}$ & $U_\mathrm{ss} \, / \unit {V}$ \\
\midrule 
%\input{data/.tex}
\end{tabular}
\end{wraptable}


\begin{figure}[h!]
\centering
\vspace*{-15pt}
%\input{data/kalibracia.tex}
\textbf{Graf 1.} Kalibrácia - závislosť počtu prekrytých maxím na posunutí meradla $x$
\end{figure}

%%%
\section*{Diskusia}

%%%
\section*{Záver}

%%%

\begin{thebibliography}{9}

\bibitem{kniha}
    \emph{Fyzikální praktikum III - Optika}, I. Pelant a kol., 2005

\bibitem{cizek}
    \emph{Úvod do praktické fyziky}, J. Čížek, seminář 11
    \url{http://physics.mff.cuni.cz/kfnt/cs/vyuka/upf/obsah.html} 11. 3. 2013

\bibitem{normal}
    \emph{Normal distribution}
    \url{http://en.wikipedia.org/wiki/Normal_distribution} 11. 3. 2013

\end{thebibliography}
\end{document}
