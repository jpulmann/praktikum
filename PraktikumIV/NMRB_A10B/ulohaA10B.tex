\documentclass[a4paper, 10pt]{article}
\usepackage[left=3cm, right=3cm, bottom=3cm, top =3cm]{geometry}
\usepackage[utf8]{inputenc}
\usepackage[slovak]{babel}
\usepackage[IL2]{fontenc}
\usepackage{amsmath}
\usepackage{amsfonts}
\usepackage{amssymb}
\usepackage{graphicx}
\usepackage{color}
\usepackage{booktabs}
\usepackage{wrapfig}
\usepackage{caption}
\usepackage{hyperref}
\usepackage[version=3]{mhchem}
\usepackage[compact]{titlesec}
\newcommand{\dd}{\ensuremath{ \mathrm{d} }}
\newcommand{\unit}[1]{\ensuremath{\, \mathrm{#1}}}
\newcommand{\di}[1]{\ensuremath{_\mathrm{#1}}}
\begin{document}
\titlespacing{\section}{0pt}{*0}{*0}
\titlespacing{\subsection}{0pt}{*0}{*0}
\titlespacing{\subsubsection}{0pt}{*0}{*0}
\title{Studium relaxací NMR v roztocích a pevné fázi}
\author{Ján Pulmann}
\date{8. 11. 2013}
\maketitle
%%%
\section*{Úlohy}
\begin{enumerate}
	\item Meranie spin-mriežkovej relaxačnej doby $T_1$ signálu NMR $^1\mathrm{H}$ v roztokoch s premennlivou koncentráciou $\mathrm {CuSO_4}$ metódou $(\pi, \pi/2)$ pulzu (\textit{inversion recovery})
	\item Meranie spin-spinovej relaxačnej doby $T_2$ signálu NMR $^1\mathrm{H}$ v roztokoch s premennlivou koncentráciou $\mathrm {CuSO_4}$ metódou spinového echa
    \item Meranie spin-mriežkovej relaxačnej doby  $T_1$ signálu NMR $^1\mathrm{H}$ vo vzorke pryže metódou inversion recovery
 \end{enumerate}

 %%%
\section*{Teória}
Princíp spin-spinovej metódy sme opísali v predchádzajúcom protokole \cite{predch}. Amplitúda signálu závisí od vzdialenosti $\pi/2$ a $\pi$ pulzu $t_w$ vzťahom z \cite{studA} 
\begin{equation}
\label{eq:teor:spinspin}
A(t_w) = A_0 \exp\left(-\frac{2t_w}{T_2}\right)\,.
\end{equation}

Pre vodu sa však v priebehu relaxácie molekuly vody tiež hýbu Brownovým pohybom. Preto sa aj po $\pi/2$-pulze a obrátení fáz momentov nesfázujú všetky momenty, keďže sa ich lokálne pole zmenilo. Ak aproximujeme zmenu poľa iba jeho lineárnym gradientom v danom mieste, dostaneme korekciu vyššieho rádu ku \ref{eq:teor:spinspin}
\begin{equation}
\label{eq:teor:spinspin2}
A(t_w) = A_0 \exp{\left( -\frac{2t_w}{T_2} - C t_w^3 \right)}\,.
\end{equation}
(Hahnov výsledok \cite{rozs}). Tento vzťah bol odvodený pre pole s axiálnou symetriou, konštanta $C$ závisí na gradiente magnetického poľa. Jeho tvar ale skúsime predpokladať aj pre všeobecné nehomogénne pole.


Relaxačnú dobu spin-mriežkovej interakcie meriame tzv. metódou \textit{inversion recovery}. Najprv úplne stočíme magnetizáciu  pomocou $\pi$-pulzu. Po čase $t$ stočíme zmenšené pole späť do roviny kolmej na $B_0$ a pozorujeme \textit{FID} signál s amplitúdou
\begin{equation}
\label{eq:teor:ir}
A(t) = A_0 \left|1 - 2\exp\left(-\frac{t}{T_1}\right)\right|\,.
\end{equation}
Taktiež pre čas 
\begin{equation}
\label{eq:teor:t0}
t_0 = T_1\ln 2
\end{equation}
nameriame nulovú intenzitu. 

Z teórie v \cite{stud} vyplýva, že v izotropných kvapalinách s nízkou viskozitou je pribline $T_1 \approx T_2$. Pre kvapaliny s vysokou viskozitou a pevné látky $T_2 \ll T_1$.

Pre čistú vodu sú obe relaxačné doby v ráde sekúnd (je veľmi ťažké takéto doby namerať, lebo aj malé prímesy majú veľký vplyv). Paramagnetické ióny vytvárajú pole o tri rády silnejšie (kvôli malej hmotnosti elektrónu). Rýchlosť relaxácie (prevrátená hodnota relaxačnej doby) závisí od štvorca tohoto poľa, teda ide o rozdiel rýchlostí 6 rádov. 

Pravdepodobnosť výskytu takéhoto silného lokálneho poľa v blízkosti iónu rastie priamo úmerne s koncentráciou, teda očakávame že aj rýchlosť relaxácie bude priamo úmerná koncentrácii.



%%%
\section*{Postup merania}
\begin{enumerate}
\item Používame vzorky roztoku modrej skalice ($\mathrm{CuSO_4}$) s geometricky rozdelenými koncentráciami 
\begin{equation}
\label{eq:postup:conc}
c_i = \frac{c_0}{2^i}, \;\; i = 0\ldots 5 \,.
\end{equation}
Pri použití metódy inversion recovery volíme časy $t$ tak, aby sme premerali celú závislosť, teda počiatky s vysokými hodnotami a okolie minima v $t_0$. Pri príliš krátkych $t$ (menej ako $2\unit{ms}$) však navyše nameriame aj spinové echo, ktoré nám zkresluje signál. Kedže nevieme v programe v praktiku určiť rozsah, z ktorého sa robí FT, volíme väčšie časy $t$. Taktiež musíme dbať na to, aby sme mali opakovaciu dobu merania oveľa väčšiu ako $T_1$, inak máme zase oslabený signál. Touto metódou odmeriame aj gumu a porovnáme hodnotu $T_1$ s nezávisle určenou hodnotou v \cite{predch}.  Počet opakovaní jedného merania volíme podľa opakovacej doby $T_0$.
\item Spin-spinovú interakciu meriame rovnako ako v \cite{predch}, tentokrát už len pre roztoky modrej skalice (úplne rovnaké meranie pre gumu sme už robili v \cite{predch})
\item Relaxačné časy $T_1$ a $T_2$ získame prekladaním nameraných závislostí amplitúdy. Pre príslušné relaxačné rýchlosti potom overíme lineárnu závislosť rýchlosti na koncentrácii. 
\end{enumerate}
\subsection*{Pomôcky}
Roztoky modrej skalice a pryž, magnet, generátory pulzov a merače, počítač s AD prevodníkom a program na spracovanie.
%%%
\section*{Výsledky merania}
Ako neistotu určenia amplitúdy opäť berieme jednu cifru, ktorá sa pri opakovaní merania hýbala, teda $0.0001\unit{arb. unit}$. Niekedy ju pre prehľadnosť nepíšeme.
\subsection*{Určenie $T_1$}
Do tabuľky 1 sme vpísali namerané závislosti amplitúdy maxima FT na vzdialenosti pulzov $t$. Tieto závislosti sme potom pre 6 vzoriek roztokov modrej skalice vykreslili do grafu 1 a prekladali sme závislosťami tvaru \ref{eq:teor:ir}. Graf má na vodorovnej osi logaritmickú škálu, kvôli prehľadnosti - väčšina hodnôt bola blízko počiatku. Výsledné parametre fitu sme aj s ich štatistickými neistotami vniesli do tabuľky 2. V grafe 2 sme závislosť relaxačnej doby spin-mriežkovej interakcie prekladali lineárnou závislosťou, podľa teórie. Pri prekladaní využijeme rozdielne presnosti nameraných hodnôt a vážime ich prevrátenou hodnotou ich disperzie.


Rovnako sme určili hodnotu $T_1$ pre gumu, pre ktorú sme ju určovali v \cite{predch} inou metódou. namerané hodnoty intenzít sú v tabuľke 3 a v grafe 3.
\begin{table}[h!]
\centering
\begin{tabular}{c|c||c|c||c|c||c|c||c|c||c|c}
\multicolumn{2}{c||}{Roztok 0} & 
\multicolumn{2}{c||}{Roztok 1} & 
\multicolumn{2}{c||}{Roztok 2} & 
\multicolumn{2}{c||}{Roztok 3} & 
\multicolumn{2}{c||}{Roztok 4} & 
\multicolumn{2}{c}{Roztok 5} 
\\
$ \frac{t}{\unit{ms}} $ & 
$ \frac{A}{\unit{arb. unit}}$ &
$ \frac{t}{\unit{ms}} $ & 
$ \frac{A}{\unit{arb. unit}}$ &
$ \frac{t}{\unit{ms}} $ & 
$ \frac{A}{\unit{arb. unit}}$ &
$ \frac{t}{\unit{ms}} $ & 
$ \frac{A}{\unit{arb. unit}}$ &
$ \frac{t}{\unit{ms}} $ & 
$ \frac{A}{\unit{arb. unit}}$ &
$ \frac{t}{\unit{ms}} $ & 
$ \frac{A}{\unit{arb. unit}}$ 
\\
\midrule 
\input{output/T1_all_tab.tex}
\end{tabular}
\newline
\vspace*{2pt}
\caption*{\textbf{ Tab. 1} Namerané hodnoty intenzity pri metóde inversion recovery v roztoku $\ce{CuSO4}$}
\end{table}

\begin{table}[h!]
\centering
\hspace*{60pt}
\begin{tabular}{c|c|c|c}
$ i $ & 
$ T_1\,/\unit{ms}$ &
$ 1/T_1\,/\unit{s^{-1}}$ &
$ A_0\,/\unit{arb. unit}$ 
\\
\midrule 
\input{output/T1c.tex}
\end{tabular}
\newline
\vspace*{2pt}
\caption*{\textbf{ Tab. 2} Parametre fitu - určenie $T_1$ roztoku $\ce{CuSO4}$}
\end{table}

\begin{figure}[h!]
\centering
\vspace*{-15pt}
\input{output/T1_ms.tex}
\caption*{\textbf{Graf 1} Namerané metódy a fity v metóde inversion recovery - závislosť $A(t)$}
\end{figure}

\begin{figure}[h!]
\centering
\vspace*{-15pt}
\input{output/T1_cfit.tex}
\textbf{Graf 2} Prekladanie závislosti relaxačného času na koncentrácii 
\end{figure}

\begin{table}[h!]
\centering
\hspace*{60pt}
\begin{tabular}{c|c}
$ t\,/\unit{ms} $ & 
$ A\,/\unit{arb. unit}$ 
\\
\midrule 
\input{output/t1guma.tex}
\end{tabular}
\newline
\vspace*{2pt}
\caption*{\textbf{ Tab. 3} Meranie metódou inversion recovery pre gumu}
\end{table}

\begin{figure}[h!]
\centering
\vspace*{-15pt}
\input{output/t1guma_graph.tex}
\caption*{\textbf{Graf 3} Meranie inversion recovery pre gumu, fit \ref{eq:teor:ir}}
\end{figure}

\subsection*{Určenie $T_2$}
Podobne ako v predchádzajúcom meraní, v tabuľke 4 sú namerané hodnoty pre všetky roztoky. V grafe 4 sú tieto hodnoty preložené závislosťou \ref{eq:teor:spinspin2}, teda až na prvé dve merania. $C$ tu totiž bolo prakticky nulové. Pretože to bol parameter navyše, výrazne to zvyšovalo štatistickú odchýľku ostatných konštánt. Merania pre vzorky 0 a 1 sme teda fitovali s fixným $C = 0$. Výsledné parametre fitu sú v tabuľke 5. V tomto grafe je prirodzene využitá zvislá logaritmická os, bez korekcie do tretieho rádu v 
\ref{eq:teor:spinspin2} by mali namerané hodnoty ležať na priamkách.  Opäť sme závislosť $T_2(c_i)$ prekladali lineárne - tieto hodnoty a fit sú v grafe 5. I tu sme jednotlivé hodnoty vážili ich neistotou.
\begin{table}[t]
\centering
\begin{tabular}{c|c||c|c||c|c||c|c||c|c||c|c}
\multicolumn{2}{c||}{Roztok 0} & 
\multicolumn{2}{c||}{Roztok 1} & 
\multicolumn{2}{c||}{Roztok 2} & 
\multicolumn{2}{c||}{Roztok 3} & 
\multicolumn{2}{c||}{Roztok 4} & 
\multicolumn{2}{c}{Roztok 5} 
\\
$ \frac{t_w}{\unit{ms}} $ & 
$ \frac{A}{\unit{arb. unit}}$ &
$ \frac{t_w}{\unit{ms}} $ & 
$ \frac{A}{\unit{arb. unit}}$ &
$ \frac{t_w}{\unit{ms}} $ & 
$ \frac{A}{\unit{arb. unit}}$ &
$ \frac{t_w}{\unit{ms}} $ & 
$ \frac{A}{\unit{arb. unit}}$ &
$ \frac{t_w}{\unit{ms}} $ & 
$ \frac{A}{\unit{arb. unit}}$ &
$ \frac{t_w}{\unit{ms}} $ & 
$ \frac{A}{\unit{arb. unit}}$ 
\\
\midrule 
\input{output/T2_all_tab.tex}
\end{tabular}
\newline
\vspace*{2pt}
\caption*{\textbf{ Tab. 4} Namerané hodnoty intenzity pri metóde spinového echa v roztoku $\ce{CuSO4}$}
\end{table}


\begin{table}[t]
\centering
\hspace*{30pt}
\begin{tabular}{c|c|c|c|c}
$ i $ & 
$ T_2\,/\unit{ms}$ &
$ 1/T_2\,/\unit{s^{-1}}$ &
$ A_0\,/\unit{arb. unit}$ &
$ C\,/\unit{10^{-6}\cdot ms^{-3}}$ 
\\
\midrule 
\input{output/T2c.tex}
\end{tabular}
\newline
\vspace*{2pt}
\caption*{\textbf{ Tab. 5} Parametre fitu - určenie $T_2$ roztoku $\ce{CuSO4}$}
\end{table}

\begin{figure}[t]
\centering
\vspace*{-15pt}
\input{output/T2_ms.tex}
\textbf{Graf 4.} Namerané metódy a fity v metóde spinového echa - závislosť $A(t_w)$
\end{figure}


\begin{figure}[t]
\centering
\vspace*{-15pt}
\input{output/T2_cfit.tex}
\textbf{Graf 5.} Závislosť relaxačnej doby $T_2$ na koncentrácii, fit   
\end{figure}
%%%


\section*{Diskusia}
V \cite{predch} sme určili pre gumu hodnotu 
$$ T_1 = (\input{../NMR_A10/output/T1.tex}) \unit{ms}\,.$$
Relatívna zmena približne $10\%$ môže byť spôsobená zmenou podmienok merania (teplota) alebo výmenou kúsku gumy.

Namerané závislosti z metódy spinového echa nám podobne ako v predchádzajúcom protokole dobre sedia s teoretickými hodnotami (graf 4.), hlavne pre veľké koncentrácie. Pre silnejšie odchýľky od závislosti \ref{eq:teor:spinspin2} už ale ani korekcia nedáva presné predpovede (hodnoty $c_4$, $c_5$). Letmým pokusom sa môžeme presvedčiť, že korekcia iba do rádu $t_w^2$ (tzn iba nahradenie $Ct_w^3\to C't_w^2$ v \ref{eq:teor:spinspin2}) dáva lepšiu zhodu s našimi nameranými hodnotami (viz graf 6 a tabuľku 6). Taktiež určené hodnoty $T_2$ sú bližšie hodnotám $T_1$. Teoretické zdôvodnenie ale ku korekcii takéhoto rádu nemáme.

Závislosť $C$ na koncentrácii je ťažko určitelná, pre malé časy sa totiž neprejavuje ($c_0$ a $c_1$ sme merali len do týchto časov). Vidíme ale klesajúci  trend, čo by súhlasilo s tým, že podľa \cite{rozs} je $C$ úmerné druhej mocnine gradientu poľa - čím menej iónov, tým menšia nehomogenita.

Pri korekcii z \cite{rozs} už relaxačné doby pre malé koncentrácie nezávisia lineárne od koncentrácie. Čiastočne to môže byť spôsobené nedokonalou zhodou našeho teoretického modelu (v tabuľke 6 sa hodnoty trochu viac podobajú tým z tabuľky 2), ale aj nečistotami vo vode. Tie majú väčší vplyv práve pri nižších koncetráciách a je možné, že inak vplývajú na $T_1$ a $T_2$. 

Pri meraní metódou inversion recovery tiež dosahujeme dobrej zhody teórie a experimentu. Lineárna závislosť relaxačnej rýchlosti od teploty je tu lepšia ako v prípade spin-spinovej interakcie (menšie neistoty parametrov fitu). 

Keďže tu meriame do vyšších časov, museli sme pri meraní meniť opakovaciu dobu $T_0$. Už pre $T_0 \approx 5t$ bol totiť rozdiel medzi takoutou a dvojnásobnou opakovacou dobou takmer dvojnásobná amplitúda peaku vo FT - pri zachovaní všetkých ostatných parametrov. Pokúsili sme sa teda voliť dostatočnú opakovaciu dobu.

V oboch prípadoch nám vyšiel absolútny člen lineárnej závislosti v rámci chyby merania nulový. Správne by mal výjsť veľmi malý (uvádza sa hodnota $1/6\unit s$). Naša presnosť ale nie je dostatočná na takéto extrapolovanie.  Napriek nepresnostiam sa ale parametre lineárnej závislosť v grafoch 2 a 5 rovnajú v rámci chyby merania.

Presnejšie meranie by sa dalo realizovať väčším počtom meraných bodov a koncentrácií. Najviac by tu pomohlo automatizovanie nastavovania generátora pulzov - samotná sekvencia merania trvá v rádoch sekúnd, no nastavovanie vyžaduje neustálu pozornosť a často niekoľko násobne viac času. Pre viac nameraných koncentrácií by sme potom vedeli posúdiť rozdiel korekcií v druhom a treťom ráde v \ref{eq:teor:spinspin}

\begin{figure}[h!]
\centering
\vspace*{-15pt}
\input{output/T2_ms2.tex}
\textbf{Graf 6} Prekladanie závislosti relaxačného času na koncentrácii s korekciou druhého rádu 
\end{figure}

\begin{table}[t]    
\centering
\hspace*{30pt}
\begin{tabular}{c|c|c|c}
$ i $ & 
$ T_2\,/\unit{ms}$ &
$ A_0\,/\unit{arb. unit}$ &
$ C'\,/\unit{10^{-3}\cdot ms^{-2}}$ 
\\
\midrule 
\input{output/T22c.tex}
\end{tabular}
\newline
\vspace*{2pt}
\caption*{\textbf{ Tab. 6} Parametre fitu - určenie $T_2$ s korekciou v druhom ráde}
\end{table}
%%%
\section*{Záver}
Použili sme metódu inversion recovery na meranie $T_1$, relaxačnej doby spin-mriežkovej interakcie (grafy 1, 2 a tabuľky 1, 2 pre modrú skalicu, graf a tabuľka 3 pre gumu). 

Porovnali sme dva spôsoby určenia $T_1$ pre gumu s relatívnou odchýľkou asi $10\%$

Namerali sme aj relaxačnú dobu spin-spinovej interakcie, tiež v závislosti na koncentrácii modrej skalice v roztoku (grafy a tabuľky 4, 5). 

Nakoniec sme otestovali prekladanie alternatívou vzťahu \ref{eq:teor:spinspin2} s korekciou iba do druhého rádu, grafy a tabuľka 6.

Ku protokolu je priložený ukážkový priebeh signálu pri metóde inversion recovery.
%%%

\begin{thebibliography}{9}


\bibitem{predch}
    \emph{Protokol ku úlohe A10}. Ján Pulmann, odovzdané 8. 11. 2013\\


\bibitem{studA}
    \emph{Študijný text ku úlohe A10} \\
    \url{http://physics.mff.cuni.cz/vyuka/zfp/_media/zadani/texty/txt_410.pdf} 5.11.2013


\bibitem{stud}
    \emph{Študijný text ku úlohe A10 B} \\
    \url{http://physics.mff.cuni.cz/vyuka/zfp/_media/zadani/texty/txt_410b.pdf} 8.11.2013

\bibitem{rozs} 
    \emph{Rozširujúci text v praktiku, časť ,,Diffusion in an Inhomogenous Magnetic Field"}

\end{thebibliography}
\end{document}
